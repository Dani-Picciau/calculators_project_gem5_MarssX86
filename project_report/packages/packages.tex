\usepackage[utf8]{inputenc}
\usepackage[T1]{fontenc}
\usepackage{lmodern}
\usepackage{graphicx}
\graphicspath{{../figures/}}
\usepackage{charter} % Font scelto per il documento
\usepackage{microtype}  % Migliora la qualità tipografica
\usepackage{titling}    % Per personalizzare il titolo
\usepackage[a4paper, margin=2cm]{geometry} % Imposta i margini
\usepackage{setspace} % Per gestire l'interlinea
\setstretch{1.1}  % interlinea di 1.1
\usepackage{enumitem}
\setlist{nolistsep} % Rimuove lo spazio tra gli elementi della lista
\usepackage{float} % Da inserire nel preambolo
\usepackage{wrapfig} % Pacchetto per posizionare testo e immagine affiancati
\usepackage{subfig} % Pacchetto per affiancare immagini
\renewcommand{\thefigure}{\arabic{figure}}
\setcounter{figure}{0}
\usepackage{xr} % Serve per poter vedere le numerazioni corrette ai riferimenti dei capitoli, nonostante stia compilando capitoli singoli
\usepackage[hidelinks]{hyperref}  % 'hidelinks' rimuove i bordi rossi
\usepackage{booktabs} % Per migliorare la qualità della tabella
\usepackage{array}    % Per una migliore gestione delle colonne
\usepackage{tabularx} %per mettere più tabbella su una riga
\usepackage{placeins}
\usepackage{colortbl} % Required for \rowcolor
\usepackage{tikz}
\usetikzlibrary{positioning, shapes.geometric}
\usepackage{amssymb}
\usepackage{changepage}

%Diego package
\usepackage{algorithm}
\usepackage{algpseudocode}
\renewcommand{\thealgorithm}{}
\floatname{algorithm}{}
\usepackage{amsmath}
\usepackage{mathtools}

\usepackage[most]{tcolorbox}
\usepackage{listings}
\usepackage{xcolor}
\usepackage[utf8]{inputenc}

\usepackage{subfiles} % Questo pacchetto ti permette di trattare ogni capitolo come un documento autonomo che "eredita" il preambolo dal file principale

% Definizione colori personalizzati
\definecolor{codebg}{RGB}{245, 245, 245} % Sfondo grigio chiaro
\definecolor{codeborder}{RGB}{200, 200, 200} % Bordo grigio più scuro
\definecolor{keywordcolor}{RGB}{0, 102, 204} % Blu per le parole chiave
\definecolor{commentcolor}{RGB}{0, 153, 0} % Verde per i commenti
\definecolor{stringcolor}{RGB}{204, 51, 0} % Rosso per le stringhe
\definecolor{typenames}{RGB}{128, 0, 128} % Viola per i tipi di dato

\lstdefinestyle{mystyle}{
    backgroundcolor=\color{codebg}, % Sfondo grigio chiaro
    frame=single, % Bordo attorno al codice
    rulecolor=\color{codeborder}, % Colore del bordo
    basicstyle=\fontfamily{pcr}\selectfont\small, % Usa il font Courier che supporta il grassetto
    keywordstyle=\color{keywordcolor}\bfseries, % Parole chiave in blu e grassetto
    commentstyle=\color{commentcolor}\itshape, % Commenti in verde e italico
    stringstyle=\color{stringcolor}, % Stringhe in rosso
    identifierstyle=\color{black}, % Identificatori normali in nero
    numberstyle=\tiny\color{gray}, % Numeri di riga in grigio
    tabsize=4, % Grandezza del tab
    showstringspaces=false, % Nasconde gli spazi nelle stringhe
    breaklines=true, % Permette di andare a capo automaticamente
    morekeywords={uint8_t, uint16_t, uint32_t, size_t}, % Aggiunta di tipi standard del C
    escapeinside={(*@}{@*)}, % Definisce i caratteri di escape per inserire comandi LaTeX
}