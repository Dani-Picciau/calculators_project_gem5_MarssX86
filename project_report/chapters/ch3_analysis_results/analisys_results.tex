\documentclass[../../main_document/main.tex]{subfiles}
\externaldocument{../../main_document/main}

\begin{document}

\section{Analisi dei dati raccolti}
I dati raccolti non verranno analizzati solo in valore assoluto, ma verranno \textbf{normalizzati rispetto alla propria baseline}. Questo approccio permette di rispondere a domande del tipo: "Quale architettura beneficia maggiormente di un aumento della cache?" o "Quale ISA risulta più efficiente a parità di risorse?". 
Il confronto finale avverrà dunque su due livelli:
\begin{enumerate}
    \item \textbf{Intra-ISA}: Confronto tra le tre build della stessa architettura per valutarne la scalabilità.
    \item \textbf{Inter-ISA}: Confronto tra i rapporti di performance delle tre architetture rispetto alle rispettive baseline, per determinare l'efficienza relativa dei set di istruzioni.
\end{enumerate}

\end{document}