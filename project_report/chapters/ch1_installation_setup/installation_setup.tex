\documentclass[../../main_document/main.tex]{subfiles}
\externaldocument{../../main_document/main}

\begin{document}
\section{Installazione e setup dei simulatori per Windows e macOS}
\subsection{Premessa fondamentale: architetture e compatibilità}
\label{ch1:par:paragrafo_1_1}
Prima di procedere, è essenziale capire che questi simulatori non sono normali applicazioni installabili con un semplice clic. Entrambi sono nati in ambito accademico per Linux, ma hanno esigenze molto diverse legate alla loro "età".

\vspace{8pt}
\noindent
Questi due simulatori architetturali (Gem5 e MarssX$86$) nascono e vengono sviluppati nativamente per sistemi Unix-like.
\begin{itemize}[leftmargin=1em]
    \item \textbf{macOS} è un sistema certificato Unix. "Parla la stessa lingua" di Linux, quindi può eseguire molti di questi programmi nativamente (direttamente dal terminale), con il solo aiuto di alcune librerie esterne.
    \item \textbf{Windows} ha un'architettura completamente diversa, per cui non può eseguire nativamente questi programmi. Per poterli eseguire è quindi necessario utilizzare delle soluzioni di \textbf{virtualizzazione o sottosistemi} per creare l'ambiente Unix necessario.
\end{itemize}

\vspace{8pt}
\noindent
Bisogna anche considerare che, anche avendo un sistema Linux (o Unix), c'è il problema delle librerie software necessarie per compilare il codice (il "Time Gap").
\begin{itemize}[leftmargin=1em]
    \item \textbf{Gem5} è un software \textbf{moderno e adattivo}, che viene continuamente sviluppato e aggiornato.\\
    Funziona senza problemi sui sistemi operativi moderni: su macOS lo si compila direttamente mentre su Windows si usa WSL con un'installazione Linux recente (Ubuntu $20.04$/$22.04$).
    \item \textbf{MarssX86} è un progetto "legacy" (molto vecchio, fermo ad $2012$ circa). Richiede compilatori (GCC vecchi) e librerie che sono state rimosse dai sistemi operativi moderni da anni.\\
    Nè macOS moderno ne WSL possono eseguirlo facilmente, dunque è necessario creare una \textbf{Macchina virtuale} (con VirtualBox) che simuli un computer del $2014$ con installato \textbf{Ubuntu $14.04$}.
\end{itemize}

\subsection{Installazione e setup del simulatore Gem5}
Per quanto riguarda l'installazione del simulatore Gem5, la soluzione varia leggermente a seconda del sistema operativo utilizzato.
\subsubsection{Installazione e setup di Gem5 su Windows}
Per installare il simulatore su Windows la soluzione migliore è \textbf{WSL} (Windows Subsystem for Linux). Questo sottosistema permette di avere un terminale Ubuntu vero e proprio integrato in Windows. Dunque, l'installazione si compone di diversi passaggi:
\begin{itemize}[leftmargin=1em]
    \item \textbf{\textit{Passo 1}: attivazione di WSL}
    \begin{enumerate}[leftmargin=1.3em]
        \item Aprire la \textbf{PowerShell} come amministratore;
        \item Digitare \cmd{\texttt{wsl ---install}} e premere invio;
        \item Attendere il download e, una volta completato, riavvare il computer;
        \item Al riavvio, seguire le istruzioni a schermo per creare \textit{username} e \textit{password} Ubuntu.
    \end{enumerate}

    \vspace{8pt}
    \item \textbf{\textit{Passo 2}: installazione delle dipendenze}\\
    Nel terminale di Ubuntu appena aperto incollare questi comandi uno alla volta:
    \begin{lstlisting}[style=mystyle, language=C, , escapeinside={(*@}{@*)}]
//Aggiorna i pacchetti
sudo apt update && sudo apt upgrade -y
    \end{lstlisting}
    \begin{lstlisting}[style=mystyle, language=C, , escapeinside={(*@}{@*)}]
//Installa compilatori, Python e librerie necessarie a Gem5
sudo apt install build-essential git m4 scons zlib1g zlib1g-dev libprotobuf-dev protobuf-compiler libprotoc-dev libgoogle-perftools-dev python3-dev python-is-python3 libboost-all-dev pkg-config -y
    \end{lstlisting}

    \vspace{8pt}
    \item \textbf{\textit{Passo 3}: scaricare e compilare Gem5}
    \begin{lstlisting}[style=mystyle, language=C, , escapeinside={(*@}{@*)}]
//Clonare il repository del progetto nel proprio workspace
git clone (*@\url{https://github.com/gem5/gem5.git}@*) 
    \end{lstlisting}
    \begin{lstlisting}[style=mystyle, language=C, , escapeinside={(*@}{@*)}]
//Entrare nella cartella "gem5"
cd workspace/gem5
    \end{lstlisting}
    \begin{lstlisting}[style=mystyle, language=C, , escapeinside={(*@}{@*)}]
//Avviare la compilazione (ci vorranno dai 10 ai 40 minuti)
//Questo comando usa tutti i core della tua CPU per fare prima
scons build/ARM/gem5.opt -j $(nproc)
    \end{lstlisting}
    \textbf{\textit{Note}}: se si vuole simulare RISC-V, bisogna sostituire \texttt{ARM} con \texttt{RISCV} nel comando \texttt{scons}.
\end{itemize}

\subsubsection{Installazione e setup di Gem5 su macOS}
Su macOS, come detto al capitolo \ref{ch1:par:paragrafo_1_1}, è possibile compilare nativamente usando il terminale, ma è necessario un gestore di pacchetti chiamato \textbf{Homebrew}.\\
Anche in questo caso possiamo suddividere l'installazione in diversi passaggi:
\begin{itemize}[leftmargin=1em]
    \item \textbf{\textit{Passo 1}: preparazione dell'ambiente}
    \begin{enumerate}[leftmargin=1.3em]
        \item Aprire il terminale;
        \item Se non si hanno, installare gli strumenti di sviluppo:  \cmd{\texttt{xcode-select ---install}};
        \item Installare Homebrew seguento le istruzioni su \url{https://brew.sh/}.
    \end{enumerate}

    \vspace{8pt}
    \item \textbf{\textit{Passo 2}: installare le dipendenze}
    \begin{lstlisting}[style=mystyle, language=C, , escapeinside={(*@}{@*)}]
brew install git scons python protobuf boost libpng hdf5 pkg-config
    \end{lstlisting}

    \vspace{8pt}
    \item \textbf{\textit{Passo 3}: scaricare e compilare Gem5}
    \begin{lstlisting}[style=mystyle, language=C, , escapeinside={(*@}{@*)}]
//Clonare il repository del progetto nel proprio workspace
git clone (*@\url{https://github.com/gem5/gem5.git}@*) 
    \end{lstlisting}
    \begin{lstlisting}[style=mystyle, language=C, , escapeinside={(*@}{@*)}]
//Entrare nella cartella "gem5"
cd workspace/gem5
    \end{lstlisting}
    \begin{lstlisting}[style=mystyle, language=C, , escapeinside={(*@}{@*)}]
//Avviare la compilazione
//Questo comando usa tutti i core della tua CPU per fare prima
scons build/ARM/gem5.opt -j $(nproc)
    \end{lstlisting}
    \textbf{\textit{Note}}: anche in questo caso, per simulare RISC-V, bisogna sostituire \texttt{ARM} con \texttt{RISCV} nel comando \texttt{scons}.
\end{itemize}

\subsection{Installazione e setup del simulatore MarssX86}
Come detto precedentemente al capitolo \ref{ch1:par:paragrafo_1_1}, MarssX$86$ è troppo vecchio per funzionare su WSL moderlo o su macOS. Esistono due strade che risolvono il problema, le quali funzionano sia per Windows che per macOS:

\subsubsection{Docker}
L'utilizzo di Docker è da prendere in considerazione nel caso in cui si voglia una maggiore rapidità, si è comodi con la riga di comando e si vuole tenere il pc pulito. Infatti, permette di scaricare un immagine di Ubuntu $14.04$ e lavorarci dentro da terminale senza installare un intero sistema operativo.
\begin{itemize}[leftmargin=1em]
    \item \textbf{\textit{Passo 1}: Installare Docker}
    \begin{enumerate}[leftmargin=1.3em]
        \item Windows/macOS: scaricare e installare Docker desktop dal sito ufficiale \url{https://www.docker.com/};
        \item Avviarlo e assicurarsi che sia attivo (controllare l'icona della balena nella barra delle applicazioni).
    \end{enumerate}

    \vspace{8pt}
    \item \textbf{\textit{Passo 2}: Creare l'ambiente (il Dockerfile)}\\
    Creare una cartella sul computer chiamata \textbf{\textit{"marss\_project"}} e al suo interno creare un file di testo chiamato  \textbf{\textit{"Dockerfile"}} in cui andrà incollato al suo interno il seguente contenuto:
    \begin{lstlisting}[style=mystyle, language=C, , escapeinside={(*@}{@*)}]
// Si usa una versione vecchia di Ubuntu compatibile con MarssX86
FROM ubuntu:14.04

// Aggiornare i repository per trovare i vecchi pacchetti
RUN apt-get update && \
    apt-get install -y build-essential git scons zlib1g-dev python g++ vim

// Impostare la cartella di lavoro
WORKDIR /root/marss

// Comando di default quando si entra nel container
CMD ["/bin/bash"]
    \end{lstlisting}

    \vspace{8pt}
    \item \textbf{\textit{Passo 3}: costruire e avviare}\\
    Aprire il terminale (o PowerShell) nella cartella dove è stato inserito il Dokerfile.
    \begin{enumerate}[leftmargin=1.3em]
        \item \textbf{Costruzione dell'immagine } tramite il comando:
        \begin{lstlisting}[style=mystyle, language=C, , escapeinside={(*@}{@*)}]
docker build -t marss_env .
        \end{lstlisting}

        \item \textbf{Ingresso nel contenitore} tramite il comando:
        \begin{lstlisting}[style=mystyle, language=C, , escapeinside={(*@}{@*)}]
docker run -it --name marss_container marss_env
        \end{lstlisting}
        Una volta dentro, il terminale "penserà" di essere nel $2014$ garantendo la possibilità di scaricare e compilare MarssX$86$ li dentro
    \end{enumerate}
\end{itemize}

\subsubsection{VirtualBox}
Invece, la VirtualBox è consigliata se si preferisce avere un'iterfaccia grafica, quindi vedere le finestre e usare un editor di testo grafico dentro la macchina virtuale, perché permette di avere un'esperienza più simile all'utilizzo di un computer fisico.
\begin{itemize}[leftmargin=1em]
    \item \textbf{\textit{Passo 1}: scaricare e installare il software}
    \begin{enumerate}[leftmargin=1.3em]
        \item Scaricare e installare \textbf{VirtualBox} al seguente indirizzo \url{https://www.virtualbox.org/};
        \item Scaricare la ISO di Ubuntu $14.04$ LTS (Trusty Tahr) al seguente indirizzo \url{https://releases.ubuntu.com/14.04/}
    \end{enumerate}

    \vspace{8pt}
    \item \textbf{\textit{Passo 2}: creazione della macchina virtuale}
    \begin{enumerate}[leftmargin=1.3em]
        \item Aprire la virtualbox e andare su "nuova";
        \item Nome: \textit{"MarssVM"}, Tipo: \textit{"Linux"}, Versione: \textit{Ubuntu (64-bit)};
        \item È necessario assegnare almeno 4GB di RAM e $2-4$ processori, fondamentali per la velocità di compilazione.
        \item Creare un disco fisico di almeno 30GB;
        \item Avviare la macchina, selezionare la ISO scaricata precedentemente e installare Ubuntu come se fosse un pc vero.
    \end{enumerate}

    \vspace{8pt}
    \item \textbf{\textit{Passo 3}: setup interno}\\
    Una volta che Ubuntu è stato installato nella virtual machine, aprire il terminale interno e lanciare i seguenti comandi, uno alla volta:
    \begin{enumerate}[leftmargin=1em]
        \item Questo comando serve per scaricare la lista aggiornata di tutto il software disponibile.
        \begin{lstlisting}[style=mystyle, language=C, , escapeinside={(*@}{@*)}]
sudo apt-get update
        \end{lstlisting}
        \item Serve per installare i \textbf{compilatori di C++} (build-essential e g++) che traducono il codice in linguaggio macchina, il \textbf{gestore dei file} (scons) che lancia in automatico tutt i file necesssari, \textbf{git} per scaricare il codice da internet e \textbf{python} perché molti script di configurazione di MarssX86 sono scritti in quel linguaggio.
        \begin{lstlisting}[style=mystyle, language=C, , escapeinside={(*@}{@*)}]
sudo apt-get install build-essential git scons zlib1g-dev python g++
        \end{lstlisting}
    \end{enumerate}
\end{itemize}
\end{document}